\documentclass[12pt]{article}
\usepackage[utf8]{inputenc}

\usepackage[margin=1in]{geometry}
\usepackage{lipsum}

\usepackage[backend=biber,style=ieee]{biblatex}
\addbibresource{sources.bib}

\usepackage{titling}
\newcommand{\subtitle}[1]{%
	\posttitle{%
		\par\end{center}
	\begin{center}\large#1\end{center}
	\vskip0.1em}}%

\title{Analytic Research Memo}
\subtitle{PEGN 430A}
\author{Tyler Singleton}
\date{17 February 2022}

\begin{document}
\maketitle

\newpage
\setlength{\parindent}{0pt}

% --- Questions Section --- %
\textbf{Questions} \\

% Question 1
\textbf{1. Provide the full name of the Treaty and its correct legal citation.} \\

Treaty: United Nations convention to combat desertification in those countries experiencing serious drought and/or desertification, particularly in Africa (UNCCD) \\

Legal Citation:
United Nations convention to combat desertification in those countries experiencing serious drought and/or desertification, particularly in Africa, with annexes, Done at Paris June 17, 1994, T.I.A.S. No 01-215 \\

% Question 2
\textbf{2. Why did you select this Treaty for this assignment?} \\

I choose this treaty because there is a wealth of information regarding its framework, action taken, environmental impact, and reporting. Additionally, climate change continues to be an important topic in politics and global discussion -- climate change was part of President Biden's platform during his 2020 campaign and a movement to form the ``Great Green Wall'' gained popularity in recent years. These facts gave me a personal interest in understanding global actions taken to reduce desertification from climate change and the United States' role.  \\

% Question 3
\textbf{3. Is your treaty a bilateral treaty or a multilateral treaty? Explain your answer and define each type of treaty. } \\

The treaty I have chosen is a multilateral treaty. The United Nations \cite{UnitedNations} have listed 114 signatories out of 197 parties. A multilateral treaty is an agreement signed by more than two countries. A bilateral treaty is, as the name suggests, an agreement signed by only two countries. \\

% Question 4
\textbf{4. Describe why the Treaty was created. This could be an historical event, environmental issue, or other reason.} \\

Per the UNCCD \cite{Convention}, the objective reason behind the treaty acts to reduce the effects of desertification and droughts for countries currently affect by these events, and to provide an internationally united front in sustainable development for such regions. The convention's official website elaborates more to highlight the early 70's drought in Sub-Saharan Africa, which killed over 200,000 people and left millions of animals dead, as the initial stimulant \cite{About}. \\

% Question 5
\textbf{5. What are the goals or purpose of the Treaty?} \\

The goals of the treaty consisted of internationally recognizing an urgent concern desertification imposes for the livelihood of those that inhabit arid, semi-arid, and dry sub-humid areas as "...[they] account for a significant portion of the Earth's land \cite{Convention}." Thus, the treaty sets forth with a purpose to combat desertification in benefit of future generations for desertification places a considerable burden on developing countries with respect to economic growth, social development, and poverty \cite{Convention}. With international corporation, this treaty aims to improve the lives of people living within the afflicted areas, restore productivity of degraded land, and work towards sustainable management of resources consistent with principles of United Nations Conference on Environment and Development, Agenda 21 \cite{About, Convention}. \\

% Question 6
\textbf{6. Describe an example of the application of the Treaty to a current (last 20 years) issue.} \\

As mentioned in my reason for why I choose this treaty, I remembered a popular discussion around the implementation of a ``Great Green Wall.'' This ambitious movement was not initiated by UNCCD, but from the African Union \cite{GreenWall}. However, the UNCCD provides significant support under Article 21 -- Financial Mechanisms \cite{Convention} -- through which it has enacted the Front Local Environnemental poir une Union Verte (FLEUVE) Project \cite{GreenWall}. The official UNCCD states this project's purpose in targeting land restoration development and employment opportunities from financial aid received by the European Commission \cite{GreenWall}. Additionally, the mechanism allowed for the Irish Government to send a 1.2 million euro grant towards supporting the Great Green Wall \cite{GreenWall} initiative. \\

% --- Reflection Questions --- %

%RQ1
\textbf{RQ1. What was especially satisfying or encouraging to you about your treaty research process and finished product (assignment submission)? } \\

I never realized the impact desertification has had around the world. There has not been a time where I lived in a situation in which water was a concern. So researching this treaty helped me realize the struggles these nations face, and it has been satisfying to see the global support these nations have received in combating desertification. Additionally, researching of the other programs initialized by the UNCCD has changed my perspective of the United Nations. I hardly hear anything regarding their actions in my general news feed, so it was awesome to find this. \\

%RQ2
\textbf{RQ2. What was challenging for you about this assignment, and how did you overcome the challenges?} \\

Finding a treaty that I was interested in, and one that had a lot of information regarding its history and application was the hardest challenge for me. I spent a lot of time looking at other treaties trying to force myself to find something interesting within them or that spurred from them. I tried using the resources provided in the library guide to help assist, but I found the U.S. Department of State's website to be the best source. Under the section of Treaties and International Agreements, they have a great list of current treaties in force which is sorted by country and issue. From this, I could much more easily find interesting treaties/agreements that were related to environmental concerns.  \\

%RQ3
\textbf{RQ3. Based on your research techniques and writing for this assignment, what goal or goals will you set for yourself for the next individual assignment?} \\

Based on my writing, I have a goal to spend more time understanding legal citations. I and how to properly use them in text. It is unusual and awkward. 

\newpage
\printbibliography


\end{document}
